\begin{abstract}
In England alone, more than 5.2 million properties are at risk of flooding. Decisions regarding future investment in defences, and operational priorities during major incidents are made with the assistance of numerical modelling, relating observations or probabilities for rainfall intensities and durations, to the likely extent of flooding. These models typically solve the shallow water equations (SWEs), or simplifications thereof, which can reproduce flooding from pluvial, fluvial, tidal, or catastrophic failure of dams or defences, with appropriate numerical schemes.

Increased availability and ubiquity of data for use in flood models is an opportunity for improved accuracy, but computational power has long limited SWE models. This thesis presents novel computational methods to perform flood simulation using SWE models with graphics processing units (GPUs) and multi-core processors, thereby significantly reducing simulation runtime. Further improvements are made by domain decomposition, with an approach allowing numerous computer systems with multiple GPU processing devices to work together for a single simulation. Simulation accuracy is not inhibited by these methods, provided 64-bit floating-point arithmetic is used.

The methods are applied successfully to flooding which is fluvial in Carlisle, pluvial in Newcastle upon Tyne, defence failure in Thamesmead, and dam failure in Malpasset. First- and second-order numerical schemes are provided, with the latter found necessary in some tests. Sensitivity of model results to spatial resolution, and parameterisations, is thoroughly evaluated. Grid refinement improves simulation accuracy, but during high-velocity events also increases sensitivity; failure to capture topographic complexity with the spatial grid risks underestimating flood extent. No discernible improvement in results was found by refinement beyond 2m.

Reliable data is scarce for short-lived pluvial flood events. Social media is considered as a potential new data source, with a new framework providing fruitful results. Future application of these methods in real-time flood forecasting systems is recommended.
\end{abstract}