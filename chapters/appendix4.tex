\chapter{iTURF simulations for Newcastle upon Tyne}

\section{Introduction}

The United Kingdom was subjected to a series of intense storms throughout 2012, bringing severe flooding and damage totalling millions of pounds. In some cases, lives were lost. Such events are not unique to the UK, with a similar situation reported across Europe. The UK Environment Agency has invested heavily in a monitoring network for major rivers, which are used as data sources in real-time hydrodynamic models. Accurate real-time observations are essential for forecasting and nowcasting during incidents, and to provide validation data for model development. However, no formalised monitoring network presently exists for surface water flooding (i.e. pluvial), which tends to be short-lived and result from convective storms which are difficult to accurately forecast. Surface water flooding from intense rainfall poses a risk to a substantial number of properties, estimated at 2.8 million \citep{Pitt2007,EnvironmentAgency2009a}. At present, a system exists to issue alerts for potential extreme rainfall; however, there is a recognised need to extrapolate from this data the specific areas at risk of flooding, which are often highly localised, sometimes to the level of individual properties \citep{Pitt2007,Golding2009}. Development of such warning systems is hampered by a lack of data, and the varied nature of different rainfall events which might ultimately result in flooding.

Flood modelling at the city scale is rarely considered feasible. The complex nature of urban environments is problematic, characterised by gradients, narrow gaps between buildings, culverted watercourses, and drainage networks of varying quality and age. Steep slopes and narrow gaps can induce supercritical flow conditions, resulting in such phenomena as hydraulic jumps, and thus requiring shock-capturing but computationally intensive models if they are to be accurately reproduced \citep{Mignot2006}. Allowing water to pass through the narrow gaps then requires high grid resolutions, typically 2m or better \citep{Schubert2012}, demanding millions of grid cells. These two factors combined mean even for the relatively short duration events typical for summer storms (i.e. 2 hours or less), model run-times are likely to be slower by an order of magnitude or more than real time. 
Improved data collection and real-time modelling of flood events allows emergency services and relevant authorities to make more-informed decisions about where they direct their attention. In some instances the areas where explicit reports of flooding are received are not those requiring the most urgent attention. Dissemination of real-time flood extent data to the public allows them to make safer choices when selecting routes for travel. Retrospectively, flood extent data has applications in determining the best location for defences, drainage upgrades, and ‘soft engineering’ strategies (i.e. warning systems, sandbags, insurance, planning constraints). 

Further development, validation and implementation of viable and accurate surface water flood warning systems requires a step change in the volume of data collected during and after flood events, and in the efficiency and capabilities of hydrodynamic modelling frameworks. Clear evidence exists that social media is increasingly used as a tool for dissemination and communication during times of crisis and natural disasters, such as during the 2011 Queensland flood and Thai flood \citep{Starbird2010,Vieweg2010,Kongthon2012,Murthy2012}; the accuracy and validity of information provided by the public through social media such as Twitter however may be questionable. A further complication is that only a small portion (approximately 1.5\% but increasing) of Tweets are precisely geotagged \citep{Crampton2013}, which is crucial information for locating and evaluating the extent of flooding. Comparison of locations geocoded from the text within Tweets against the actual location of the user from geotags suggests even when Tweets are geotagged, this data can rarely be considered reliable for inferring flooded locations \citep{Leetaru2013}. Clearly, an alternative approach is required.

This paper makes a contribution to both understanding the geographic components of Twitter data, and integration thereof with real-time flood modelling. We contribute to on-going discussions regarding the possibilities and challenges of actively engaging with the public through social media for hazard and risk management.  The framework demonstrates that social media provides an excellent source of data, and that its utility may be further enhanced when coupled with efficient graphics processing unit (GPU) accelerated real-time high-resolution hydrodynamic modelling. Some limitations are also identified, insofar as capturing the spatial and temporal variations in rainfall intensity, and correctly interpreting the meaning of social media messages. 

Snipped.

Snipped.

\section*{Acknowledgements}
The authors are grateful for the support of the Engineering and Physical Sciences Research Council (EPSRC) in funding this work through the Sustainable Society Network+ (EP/K003593/1) and the GPU devices used (EP/K031678/1). Gratitude is also expressed to Darren Cummings for the photograph in \ref{NclSM-Example-Supercritical}.

\section{Fluvial flooding in the Eden}

Some text.
