\begin{publications}
This thesis contains both unpublished and refactored versions of published work. Whilst the published works were collaborative, the words herein are original, and below is a list of the publications which resulted from this work, and a brief statement of originality for those elements included.

\begin{itemize}
	\item\raggedright Liang, Q., \textbf{Smith, L.S.}, and Xilin, X. (2016) New prospects for computational hydraulics by leveraging high-performance heterogeneous computing techniques, \textit{Journal of Hydrodynamics}, Ser. B 28(6), 977-985. \\[1ex]
	This paper contains some of the figures found in Chapters \ref{chapter:Decomposition} and \ref{chapter:ScaleEffects}, and description of the modelling applied for pluvial flooding in Newcastle upon Tyne. The relevant sections, figures, and modelling undertaken for were written by the author of this thesis, and the other sections discussing the numerical methods do not appear herein.
	\\[2ex]
	
	\item\raggedright Liang, Q. and \textbf{Smith, L.S.} (2015) A high-performance integrated hydrodynamic modelling system for urban flood simulations, \textit{Journal of Hydroinformatics}, 17(4), 518-533. \\[1ex]
	The results in this paper for simulations of hypothetical events in Glasgow and Thamesmead appear in Chapters \ref{chapter:NumericalValidation} and \ref{chapter:Fluvial}. The modelling, description thereof, and discussion of results in this case are the work of the author of this thesis. Some of the detail and figures describing the software in the methodology, are the same as those originally published in 2013 by the author of the thesis.
	\\[2ex]
	
	\item\raggedright \textbf{Smith, L.S.}, Liang, Q., James, P., and Lin, W. (2015) Assessing the utility of social media as a data source for flood risk management using a real-time modelling framework, \textit{Journal of Flood Risk Management}, doi:\href{http://dx.doi.org/10.1111/jfr3.12154}{10.1111/jfr3.12154} \\[1ex]
	This majority of this work is reproduced in Chapter \ref{chapter:SoftData}. The words, simulation results, and discussion thereof are work by the author of this thesis, and were improved by consultation with the co-authors, who were co-investigators on the project.
	\\[2ex]
	
	\item\raggedright \textbf{Smith, L.S.}, Liang, Q., Quinn, P.F. (2015) Towards a hydrodynamic modelling framework appropriate for applications in urban flood assessment and mitigation using heterogeneous computing, \textit{Urban Water Journal}, 12(1), 67-78. \\[1ex]
	This paper is an improvement upon a conference paper authored and presented by the author of this thesis. It presents the software application to the Carlisle 2005 flooding, which appears in Chapter \ref{chapter:Fluvial}. The words in the paper are those of the first author, improved by consultation and review by the co-authors. Further advice and data provided by Jeffrey Neal of the University of Bristol is acknowledged.
	\\[2ex]
	
	\item\raggedright \textbf{Smith, L.S.} and Liang, Q. (2013) Towards a generalised GPU/CPU shallow-flow modelling tool, \textit{Computers \& Fluids}, 88, 334-343. \\[1ex]
	A portion of this publication is reproduced in Chapter \ref{chapter:NumericalMethods}, including figures, discussion and results for the Malpasset flood in Chapter \ref{chapter:NumericalValidation}. The software authorship, figures, and modelling applicable is the work of the author of this thesis.
	\\[2ex]
	
	\item\raggedright Rashid, A.A., Liang, Q., Dawson, R.J., and \textbf{Smith, L.S.} (2016) Calibrating a High-Performance Hydrodynamic Model for Broad-Scale Flood Simulation: Application to Thames Estuary, London, UK. \textit{Procedia Engineering}, 154, 967-974. \\[1ex]
	No work relating to this publication is used or discussed in this thesis, however the authors collaborated with the author of this thesis, and used the software resulting from this work, constituting one of the largest and most-detailed applications.
	\\[2ex]
\end{itemize}

The author of this thesis has contributed to other publications, which do not form any part of the work in this thesis.
\end{publications}