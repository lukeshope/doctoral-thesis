\chapter{First-order FV with Glasgow and Thamesmead}

\section{Introduction}

The past year of 2012 has seen the UK and numerous regions around the world subjected to an unusually wet summer, causing severe flash flooding. In July, August and September, many places in the UK, including Wales, Cornwall, Devon, North Somerset, North and West Yorkshire, Newcastle and the Scottish Borders, suffered flash flooding that caused substantial damage to property and major disruption to transport networks. Outside of the UK, a devastating flood killed 144 people in the Krasnodar region of Russia in July; Beijing in China, Uttarakhard in India and Manilla in the Phillipines have also been struck since. Most commonly associated with torrential rainfall, these flash floods are characterised by a sudden rise in river and subsequent floodplain inundation, and high-velocity overland flow following a rapid catchment response to the intense rainfall. Numerous processes related to the catchment response including bore formation (so-called 'walls of water') remain poorly understood but numerical modelling provides a means through which these events may be reproduced. Reliable simulation of these violent and unpredictable natural events demands a shock-capturing hydrodynamic model, generally beyond the capabilities of hydrological and simplified hydraulic modelling. However, the high computational burden associated with full hydrodynamic models has restricted their wider application to small spatial extents and short duration events. Most of the existing shock-capturing hydrodynamic models are not able to provide efficient and high-resolution simulations for large-scale flash flood events. Particularly, most of the aforementioned flash floods occur in urban areas, where high-resolution simulation is essential in order to resolve the complex urban topographic features consisting of buildings, streets and embankments in order to provide reliable numerical predictions. This poses a great challenge to existing two-dimensional hydrodynamic flood modelling tools, which are generally computationally demanding. This work therefore presents a new High-Performance Integrated hydrodynamic Modelling System (Hi-PIMS) for efficient high-resolution urban flood modelling.

-- SNIP --

By creating a refined mesh only in those areas of interests, dynamic grid adaption provides an effective means to relax the high computational burden inherent in the full dynamic inundation models. While retaining the robustness and complexity of the numerical approach, dynamic grid adaptation however presents new problems if mass conservation is to be achieved. It is unlikely to increase the timestep for cases where the highest level of refinement is concentrated on the most complex flow dynamics and highest velocities or free-surface gradients \citep[see descriptions of refinement criteria in][]{Liang2008a,Kubatko2009}.

Rather than creating high-resolution mesh to directly capture small-scale topographic or flow features as used in the adaptive mesh methods, different sub-grid parameterisation techniques have also been proposed to integrate high-resolution topographic features into flood models to enable more accurate and efficient coarse-resolution simulations \citep[e.g.][]{Soares-Frazao2008,Guinot2012,Schubert2012,Chen2012}. Most of these models are essentially based on rigorous reformulation of full dynamic shallow water equations to effectively reproduce complex urban topography \citep[e.g.][]{Soares-Frazao2008,Guinot2012,Schubert2012}. \citet{Soares-Frazao2008} introduced a new shallow flow model with porosity to account for the reduction in storage due to sub-grid topographic features. The performance of the porosity model was compared with that of a refined mesh model explicitly reflecting sub-grid scale urban structures and a more classical approach of raising local bed roughness. While able to reproduce the mean characteristics of the urban flood waves at a much lower computational cost than the refined mesh simulations, the porosity model was unable to accurately predict the formulation and propagation of certain localised wave features, e.g. reflected bores. In another study, \citet{Schubert2012} investigated various approaches of representing sub-grid topographic features in simulating a dam-break flood in an urban area and concluded that only those methods taking into account of building geometries can capture building-scale variability in the velocity field. They also indicated the benefit of using high-resolution simulation to explicitly represent buildings and road structures if run-time execution costs were not a major concern \citep[see also][]{Gallegos2009}. The issue of high-computational cost may be resolved by exploring recent developments in computational hardware.

-- SNIP --

This work aims to present a high-performance hydrodynamic model for simulating different types of urban inundation processes including those induced flash floods, which generally involves rapidly-varying transcritical flows that should be reliably resolved by a robust shock-capturing shallow water model. The new computational power provided by recent developments in GPU computing is explored by adopting the model framework as introduced in \citet{Smith2013}, which solves the 2D shallow water equations using a 2nd-order accurate shock-capturing finite-volume Godunov-type scheme. The model can take advantage of either CPUs or GPUs with a single codebase by leveraging OpenCL. To further improve the computational efficiency of the modelling framework, a 1st-order accurate scheme is implemented, due to the fact that a 2nd-order accurate numerical scheme may not always be superior in real-world applications \citep{Zhang2013}. Extra source and sink terms are also included in the new implementation to better describe the urban rainfall-runoff processes. With Hi-PIMS, the focus of this work is to demonstrate that high-resolution large-scale urban inundation modelling can be realised at an affordable computational cost. Application to a standard benchmark test in Glasgow and a hypothetical case at Thamesmead reaffirms its efficiency and suitability. Results are presented for different vendors' devices.

\section{Finite-volume Godunov-type shallow flow solver}

Snipped.

\section{GPU-accelerated framework}

Snipped.

\section{Results and discussion}

The aforementioned software has been applied to inundation simulations in Glasgow and Thamesmead, both in the United Kingdom. The first represents a standard test-case to allow comparison with commercially-available hydraulic modelling packages. The latter represents a much larger-scale test that would be burdensomely slow without GPU acceleration. The Courant number is 0.5 for both cases. Both cases are simulated using an Intel Xeon E5-2609 CPU device \citep{Intel_12}, AMD FirePro V7800 GPU \citep{AdvancedMicroDevicesInc2010a}, and NVIDIA Tesla M2075 GPU \citep{NVIDIACorporation2011}.

\subsection{Glasgow}

Snipped.

\subsection{Thamesmead}

Snipped.

\section{Conclusions}

We have presented a GPU-accelerated shallow flow model for urban flood modelling. Benefiting from a shock-capturing capability as a result of implementing a first-order accurate finite-volume Godunov-type scheme, the model is able to reproduce a wide range of complex flow hydrodynamics including hydraulic jump-like transcritical flow with discontinuities. It is therefore well-suited to urban flood modelling where local complex flow hydrodynamics may occur due to flood waves interacting with complex structures and topographic features. The model also ensures non-negative water depth through a reconstruction technique, thereby allowing robust simulation of realistic flood events with wetting and drying without causing numerical instability. 

In the past, due to high computational expense, it has been challenging to apply such a sophisticated fully-2D model for high-resolution simulations across large spatial extents. With the assistance of OpenCL and increased support for the standard by mainstream vendors, the presented modelling package enables simulations on a range of devices including multi-core CPUs and GPUs, which allows us to leverage the benefits of heterogeneous computing. It has been observed from the simulation results that significant speed-up is achievable even with inexpensive desktop GPUs not designed for scientific use. 

With this new GPU-accelerated modelling package, high-resolution flood modelling with shock-capturing finite-volume schemes becomes feasible. The results for Thamesmead reveal that high-resolution flood modelling predicts markedly different results to coarse resolutions, justifying the need to comprehensively depict complex urban topographies. Preliminary work has also been conducted to compare the use of 32- and 64-bit floating-point arithmetic. Using 32-bit arithmetic, significant errors in mass conservation occur and computational efficiency is hindered by small time steps in the Glasgow test, occurring due to the extremely small water depths following rainfall and domain wetting and drying. Therefore further research is required to conclude whether 32-bit arithmetic is sufficiently accurate for flood simulations.
